\documentclass{book}

%%%%%%%%%%%%%%%%%%%%%%%%%%%%%%%%%%%%%%%%%%%%%%%%%%%%%%%%%%%%%%%%%%%%%%%%%%
\usepackage{graphicx} % Required for inserting images
\usepackage{amsmath} % basic math stuff, also aligned environment 
\usepackage{amsfonts} % defeines blackboard bold fonts for \Z, \R
\usepackage{tikz} % Drawing environment 
\usepackage{amsthm} % AMS theorem defining stuff
\usepackage{amssymb} % good looking empty set
\usepackage{leftindex} % let superscript

% Formating stuff
\usepackage{hyphenat} % forbid use of hyphen
%complete elimination of hyphen 
\righthyphenmin = 100
\lefthyphenmin = 100
%取消段前空格
\setlength{\parindent}{0pt}

% Makes section headings and cross references act like links
\usepackage[colorlinks,unicode]{hyperref}

%%%%%%%%%%%%%%%%%%%%%%%%%%%%%%%%%%%%%%%%%%%%%%%%%%%%%%%%%%%%%%%%%%%%%%%%%%
% Create simple shortcut commands.  Almost everyone who uses LaTeX
% creates commands like thes.
\newcommand{\R}{\mathbb{R}}
\newcommand{\Q}{\mathbb{Q}}
\newcommand{\Z}{\mathbb{Z}}
\newcommand{\N}{\mathbb{N}}
\newcommand{\C}{\mathbb{C}}

% this is to highlight words that are being defined and enter.
\newcommand{\define}[1]{\textbf{#1}}

% Symmetric inclusion symbol 
\newcommand{\syin}{$\subset$\kern-0.48em\raisebox{.20ex}{\tiny S}$\,$}
%\kern 左右移动(-为向左),\raisebox 上下移动(-为向下)
% Note this symbol does not work in math mode.

% diam
\newcommand{\diam}[1]{\textrm{diam}( #1 )}
% inner product 
\newcommand{\inner}[1]{\langle #1 \rangle}
% norm 
\newcommand{\norm}[1]{\lVert #1 \rVert}
% dist
\newcommand{\dist}[1]{\textrm{dist}( #1 )}
% absolute value 
\newcommand{\abs}[1]{\lvert #1 \rvert }
% Cayley graph 
\newcommand{\Cay}[1]{\textrm{Cay}( #1 )}
% Cos graph
\newcommand{\Cos}[1]{\textrm{Cos}( #1 )}
% Automorphism group 
\newcommand{\Aut}[1]{\textrm{Aut}( #1 )}

%%%%%%%%%%%%%%%%%%%%%%%%%%%%%%%%%%%%%%%%%%%%%%%%%%%%%%%%%%%%%%%%%%%%%%%%%%%%
% The next page or two of stuff is devoted to creating and
% manipulating the environments for lemma, proof, example, etc.
% The first part is pretty simple, and almost everyone has commands
% like this in every paper, article, book, etc.  
% 
% We start with things like lemmas, theorems, etc.

%注释
%\newtheorem{env name}{text}[parent counter]
%\newtheorem{env name}[shared counter]{text}
%parent counter: numbering will resteart whenever that sectional level is encountered
%shared counter: if specificed, numbering will progress sequentially for 
%all theorem element using this counter

\theoremstyle{definition} 
%adds extra space above and below, but sets the text in roman
%recommended for definitions, conditions, problems, and examples

\newtheorem{lemma}{Lemma}[chapter]
\newtheorem{proposition}[lemma]{Proposition}
\newtheorem{corollary}[lemma]{Corollary}
\newtheorem{definition}[lemma]{Definition}

%自定义新的theorem environment
\newtheoremstyle{remarkstyle}
    {\topsep}%   space above, empty for `usual value'
    {\topsep}%   space below
    {}%          body font, input \itshape for italian
    {}%          indent amount 
    {}%          thm head font 
    {}%          Punctuation after thm head
    {\newline}%  Space after thm head: \newline = linebreak
    {}%          Thm head spec
\theoremstyle{remarkstyle}
\newtheorem*{remark}{Remark}%[chapter] 
\newtheorem*{example}{Example}%[chapter]
\newtheorem*{myproof}{Proof}%[chapter]
%this formating (* and %[chapter]) elminates numbering for theorem environment

%theorem之间的空间是手动调的,在最后加个\newline或者item[]

%%%%%%%%%%%%%%%%%%%%%%%%%%%%%%%%%%%%%%%%%%%%%%%%%%%%%%%%%%%%%%%%%%%%%%%%%%%%%%
%%%%%%%%%%%%%%%%%%%%%%%%%%%%%%%%%%%%%%%%%%%%%%%%%%%%%%%%%%%%%%%%%%%%%%%%%%%%%%
%%%%%%%%%%%%%%%%%%%%%%%%%%%%%%%%%%%%%%%%%%%%%%%%%%%%%%%%%%%%%%%%%%%%%%%%%%%%%%
%%%%%%%%%%%%%%%%%%%%%%%%%%%%%%%%%%%%%%%%%%%%%%%%%%%%%%%%%%%%%%%%%%%%%%%%%%%%%%
% The stuff that follows is a bit tricky, and is well more than most
% people do with LaTeX.  The basic idea is that I want to be able to
% show only examples, or only definitions, or hide only proofs, etc.  


\makeatletter
% the following key values give the set up for hiding certain
% environemnts. Note that "true" is just a dummy value.  It would work
% equally well with "foo".  Each one of these redefines the outer
% environment for an atom to turn the whole thing into a hidden
% comment.  Note that the environments theorem, example, etc. do not
% have to be written in any special way for hide and show to work.
% They get clobbered by hide.  
  \define@key{hide}{lemma}[true]{\renewenvironment{lemma}{\comment}{\endcomment}}
  \define@key{hide}{comments}[true]{\renewenvironment{comments}{\comment}{\endcomment}}
  \define@key{hide}{corollary}[true]{\renewenvironment{corollary}{\comment}{\endcomment}}
  \define@key{hide}{definition}[true]{\renewenvironment{definition}{\comment}{\endcomment}}
  \define@key{hide}{example}[true]{\renewenvironment{example}{\comment}{\endcomment}}
  \define@key{hide}{proposition}[true]{\renewenvironment{proposition}{\comment}{\endcomment}}
  \define@key{hide}{remark}[true]{\renewenvironment{remark}{\comment}{\endcomment}}
    \define@key{hide}{myproof}[true]{\renewenvironment{myproof}{\comment}{\endcomment}}
% Now this command can be given a comma separated list of names to
% hide.  Thus, \HideEnvirons{ example, proof} would hide all examples
% and proofs.
\newcommand{\HideEnvirons}[1]{\setkeys{hide}{#1}}


% The \ShowEnvirons command works like this: if it's argument is foo,
% it redefines the *hide* family key for foo to do nothing.  Then, it
% issues a \HideEnvirons command containing a list of all the atom
% types.  So, everything is hidden *unless* the hide-family key has
% been redefinied.
  \define@key{show}{lemma}[true]{\define@key{hide}{lemma}{}}
  \define@key{show}{comments}[true]{\define@key{hide}{comments}{}}
  \define@key{show}{corollary}[true]{\define@key{hide}{corollary}{}}
  \define@key{show}{definition}[true]{\define@key{hide}{definition}{}}
  \define@key{show}{example}[true]{\define@key{hide}{example}{}}
  \define@key{show}{proposition}[true]{\define@key{hide}{proposition}{}} \define@key{show}{remark}[true]{\define@key{hide}{remark}{}}
  \define@key{show}{myproof}[true]{\define@key{hide}{myproof}{}}
% This command can be given a comma separated list of names to show,
% and only these will be shown.  Thus,
% \ShowEnvirons{example,definition} will show only examples and definitions.
\newcommand{\ShowEnvirons}[1]
{\setkeys{show}{#1}\HideEnvirons{%
    comments,
    corollary,
    definition,
    example,
    proposition,
    lemma,
    remark,
    myproof
  }}
\makeatother

% Comment out the next line to see the full text
\ShowEnvirons{definition, proposition, remark, example, lemma, myproof, corollary}


%%%%%%%%%%%%%%%%%%%%%%%%%%%%%%%%%%%%%%%%%%%%%%%%%%%%%%%%%%%%%%%%%%%%%%%%%%%%%%
\begin{document}

\setcounter{chapter}{3}
\chapter{Diameters of Cayley Graphs and Expander Families}
% Tips facing very long chapter titles 
% \chapter[medium-length title for Table of contents, if wanted]{full title name}
% \chaptermark{short title for running headers}
\chaptermark{Graph Diameters}



Good communication networks = Messages spread quickly = Small diameters \newline
In this chapter, we show necessary conditions for expander family related to diameters (namely logarithmic diameter), and these conditions can be further related to group structures. There is no equivalent conditions: we illustrate this in the end of the chapter by constructing an "almost" expander family example. 

\section{Expander Families have Logarithmic Diameter}
\begin{definition}
    Graph $X$, vertex $v$ of $X$, non-negative integer $r$. Define the \define{closed ball} of radius r centred at v as $B_{r}[v]$, the \define{sphere} of radius r centred at v as $S_{r}[v]$.
    \begin{align*}
        B_{r}[v] &= \{ w \in V_{X} \, \vert \, \dist{v,w} \le r\} \\
        S_{r}[v] &= \{ w \in V_{X} \, \vert \, \dist{v,w} = r\}
    \end{align*} 
\end{definition}
\begin{remark}
    \begin{itemize}
        \item[]
        \item This terminology resembles that of metric spaces. Recall that dist is a metric. 
        \item By definition, $B_{r}[v], S_{r}[v] \subset V_{X}$ for any $r$.
        \item[] 
    \end{itemize}
\end{remark}


\begin{lemma}
    $\{X_{n}\}$ is a sequence of $d$-regular graphs with $\abs{X_{n}} \rightarrow \infty $ and $d \ge 3$. Then, $\diam{X}$ grows at least logarithmically. Equivalently, $\diam{X_{n}} = \Omega(\ln X_{n})$. 
\end{lemma}
\begin{myproof}
    \begin{itemize}
        \item[]
        \item Consider the case with $\diam{X} \ge 3$. Let $v$ be a vertex of $X$. Suffice to show $\diam{X} \ge \log_{d}\abs{X} $
        \item Note $\abs{S_{0}[v]}=1$ and $\abs{S_{1}[v]}\le d$ (strict $<$ for loops). 
        \item If $j \ge 2$, then for any vertex $w$ of $S_{j}[v] $, at least one edge incident ot $w$ is also incident to a vertex in $S_{j-1}[v] $. Hence, no more than $d-1$ of these edges are incident to vertices in $S_{j+1}[v] $. Thus, $\abs{S_{j+1}[v]}\le (d-1)\abs{S_{j}[v]}$, 
        \item Induction, $\abs{S_{j}[v]} \le d(d-1)^{j-1}$ (since $\abs{S_{1}[v]}\le d$).
        \item For any $r$, $B_{r}[v] $ is the disjoint union of $S_{0}[v], S_{1}[v], \dots, S_{r}[v] $, thus $$\abs{B_{r}[v]} \le 1+d \left( \sum_{j=0}^{r-1} (d-1)^{j} \right) $$. 
        \item RHS is a polynomial in $d$ of degree $r$, thus controlled by $d^{r}$. Claim: for $r \ge 3$, $\abs{B_{r}[v]} \le d^{r} $. 
        \item Note for $r \ge 3$, $0 \le d^{2}-3d+1 $, thus $(d-1)^{3} \le d^{2}(d-2) $, thus $d(d-1)^{r} = (d-1)^{r-3}(d-1)^{3} \le d^{r-3}d^{2}(d-2) = d^{r-1}(d-2) $, thus $d(d-1)^{r}-2 \le d(d-1)^{r} \le d^{r}(d-2) $. 
        \item Hence, $$ \abs{B_{r}[v]} \le 1+d \left( \sum_{j=0}^{r-1} (d-1)^{j} \right) = 1 + d \left[ \frac{(d-1)^{r}-1}{d-2} \right] \le d^{r} $$
        \item Let $r = \diam{X} $, thus $\abs{X}=\abs{B_{r}[v]} \le d^{r} $, so $\diam{X} \ge \log_{d}\abs{X} $
    \end{itemize}
\end{myproof}
\begin{remark}
    Logarithmic diameter growth is the slowest possible case. Our next goal is show expander family must achieve this slowest growth case. \newline 
\end{remark}


\begin{lemma}
    Connnected finite graph $X$. Let $a >1$. Suppose that for any vertex $v$ of $X$, $\abs{B_{r-1}[v]} \le \frac{1}{2}\abs{X}$ always implies that $\abs{B_{r}[v]}\ge a^{r} $. Then $$\diam{X}\le \left( \frac{2}{\ln a} \right) \ln \abs{X}  $$
\end{lemma}
\begin{myproof}
    \begin{itemize}
        \item[]
        \item $w_{1}, w_{2} $ two vertices of $X$. Let $r_{1} $ be the smallest non-negative integer s.t. $\abs{B_{r_{1}}[w_{1}]} > \frac{1}{2}\abs{X}$. Such $r$ must exist since $X$ is connected thus having finite diameter. 
        \item Then by assumption $\abs{B_{r_{1}}[w_{1}]} \ge a^{r_{1}} $, since $r_{1} -1 < r_{1} $ and therefore $\abs{B_{r_{1}-1}[w_{1}]} \le \frac{1}{2}\abs{X} $. 
        \item Similarly, let $r_{2} $ be the smallest non-negative integer s.t. $\abs{B_{r_{2}}[w_{2}]} > \frac{1}{2}\abs{X}$, thus $\abs{B_{r_{2}}[w_{2}]} \ge a^{r_{2}} $
        \item Note $\abs{B_{r_{1}}[w_{1}]}+\abs{B_{r_{2}}[w_{2}]}> \abs{X}$, so $B_{r_{1}}[w_{1}] \cap B_{r_{2}}[w_{2}] \ne \varnothing $. Let $w_{3} \in B_{r_{1}}[w_{1}] \cap B_{r_{2}}[w_{2}]  $. 
        \item Then $\dist{w_{1}, w_{3}}\le r_{1}, \; \dist{w_{2}, w_{3}} \le r_{2} $, so \begin{align*}
        \dist{w_{1}, w_{2}} &\le \dist{w_{1}, w_{3}} + \dist{w_{3}, w_{2}} \\
        &\le \log_{a}\abs{B_{r_{1}}[w_{1}]} + \log_{a}\abs{B_{r_{2}}[w_{2}]} \\
        &\le \left( \frac{2}{\ln a} \right) \ln\abs{X}
        \end{align*}
        $(B_{r_{1}}[w_{1}], B_{r_{2}}[w_{2}]$ are all subset of $V_{X}) $
        \item Since this holds for arbitrary $w_{1}, w_{2} $, done. 
        \item[] 
    \end{itemize}
\end{myproof}


\begin{proposition}
    $X$ is a connected $d$-regular graph. Let $C = 1+\frac{h(X)}{d} $. Then $$
    \diam{X} \le \left( \frac{2}{\ln C} \right) \ln\abs{X}
    $$
\end{proposition}
\begin{remark}
    If $h(X_{n} )$ is bounded away from 0 (hence an expander family), then $\diam{X_{n} }$ grows at most logarithmically as a function of $\abs{X_{n} }$. From Lemma 4.2, we know this is the slowest growth possible.
\end{remark}
\begin{myproof}
    \begin{itemize}
        \item[]
        \item Let $v$ be a vertex. Suppose $\abs{B_{r-1}[v]} \le \frac{1}{2}\abs{X}$, then by definition of isoperimetric constant, $$
        \abs{\partial B_{r-1}[v]} \ge h(X)\abs{B_{r-1}[v]}
        $$
        \item Any edge in $\partial B_{r-1}[v]$ must be incident to a vertex in $S_{r}[v] $. Since $X$ is $d$-regular, we have $$
        \abs{S_{r}[v]} \ge \frac{\partial B_{r-1}[v]}{d} \ge \frac{h(X)}{d}\abs{B_{r-1}[v]}
        $$
        \item Note $B_{r}[v] $ is the disjoint union of $B_{r-1}[v]$ and $S_{r}[v]$, thus $$
        \abs{B_{r}[v]} = \abs{B_{r-1}[v]} + \abs{S_{r}[v]} \ge \abs{B_{r-1}[v]} + \frac{h(X)}{d}\abs{B_{r-1}[v]} = C\abs{B_{r-1}[v]}. 
        $$
        \item By induction, $\abs{B_{r}[v]} \ge C^{r} $, which is implied by $\abs{B_{r-1}[v]} \le \frac{1}{2}\abs{X}$. Hence, by Lemma 4.3, substitute $a$ by $C$, done. 
        \item[]
    \end{itemize}
\end{myproof}


\begin{definition}
    $\{X_{n} \}$ has logarithmic diameter if $\diam{X_{n}} = O(\ln\abs{X_{n}}) $ \newline
\end{definition}


\begin{corollary}
    Non-negative integer $d$. If $\{X_{n}\} $ is a family of $d$-regular expanders, then $\{X_{n}\} $ has logarithmic diameter. \newline
\end{corollary}


\section{Diameters of Cayley Graphs}
\begin{definition}
    Let $\{G_{n} \} $ be a sequence of finite groups. We say $\{G_{n} \}$ has logarithmic diameter if for some non-negative integer $d$, there exists $\{\Gamma_{n} \}$ s.t. $\Gamma_{n} \syin G_{n} $ and $\abs{\Gamma}=d$ for each n, so the sequence of Cayley graphs $\{ \Cay{G_{n}, \Gamma_{n}} \}$ has logarithmic diameter. \newline
\end{definition}


\begin{definition}
    Let $\Gamma$ be a set, $n$ be a positive integer. Then a \define{word} of length $n$ in $\Gamma$ is an element of the Cartesian product $\Gamma \times \dots \times \Gamma = \Gamma^{n} $. If $\Gamma \subset G $ for some group G and $w = (w_{1}, \dots, w_{n}) $ is a word in $\Gamma$, then $w$ evaluates to $g$ in $G$ if $g = w_{1}\dots w_{n} $.
\end{definition}
\begin{remark}
    Clearly, an element in $G$ can be expressed as different words. It's also possible that for certain $\Gamma$, some elements in $G$ cannot be expressed as any word. An example is $G=\Z_{10}, \; \Gamma = \{0,2,4,6,8\} $ \newline
\end{remark}


\begin{definition}
    Group $G$, $\Gamma \subset G$, $g \in G$ and can be expressed as a word in $\Gamma$. The \define{word norm} of $g$ in $\Gamma$ is the minimal length of any word in $\Gamma$ that evaluates to $g$. By convention, identity element has word norm 0. \newline
\end{definition}


\begin{proposition}
    Finite group $G$. Let $\Gamma$ symmetric in $G$ and $X = \Cay{G, \Gamma}$ Then:
    \begin{itemize}
        \item[1] $X$ is connected iff every element of $G$ can be expressed as a word in $\Gamma$.
        \item[2] If $a,b\in G$ and there is a walk in $X$ from $a$ to $b$, then the distance from $a$ to $b$ is the word norm of $a^{-1}b $ in $\Gamma$.
        \item[3] The diameter of $X$ equals the maximum of the word norms in $\Gamma$ of elements of $G$.  
    \end{itemize}
\end{proposition}
\begin{myproof}
    \begin{itemize}
        \item[]
        \item[1] Equivalent to say $\Gamma$ generates $G$.
        \item[2] 
        \begin{itemize}
            \item Let $a = g_{0}, b=g_{n} $, and $(g_{0}, g_{1}, \dots, g_{n}) $ be a walk of length $n$ in $X$ from $a$ to $b$. 
            \item Let $\gamma_{j} = g^{-1}_{j-1}g_{j} $ for $j = 1, \dots, n$. Then since there is an edge from $g_{j-1} $ to $g_{j} $, we have $\gamma_{j} \in \Gamma $ and $(\gamma_{1}, \dots, \gamma_{n}) $ is a word of length $n$ in $\Gamma$ that evaluates to $a^{-1}b $. 
            \item Reversing this procedure, we see that every word of length $n$ in $\Gamma$ that evaluates to $a^{-1}b $ corresponds to a path of length $n$ in $X$ from $a$ to $b$. 
            \item Thus, the distance from $a$ to $b$ is the minimal length of all walks from $a$ to $b$, which equals to the minimal length of all words hence the word norm of $a^{-1}b$.
        \end{itemize}
        \item[3] Natural implication of (2).
        \item[] 
    \end{itemize}
\end{myproof}


\section{Abelian Groups Never Yield Expander Families}
\begin{lemma}
    The number of solutions to the equation $a_{1} + \dots + a_{n} = k $, where $a_{i} $ are non-negative integers, is $$C^{k}_{n+k-1} $$
\end{lemma}
\begin{myproof}
    Equivalently, consider partitioning $k+n$ balls into $n$ boxes, and each box has at least one ball. You can place $n-1$ boards between these balls to make the partition, and there are $k+n-1$ spaces where you can place boards. \newline
\end{myproof}

\begin{lemma}
    If $a,b \in \N, \; b\le a $, then $$
    C^{b}_{a} \le (a-b+1)^{b}
    $$ 
\end{lemma}
\begin{myproof}
    Observe that if $0 < q \le p$, then $\frac{p+1}{q+1} \le \frac{p}{q} $. Hence 
    $$
    \frac{a}{b} \le \frac{a-1}{b-1} \le \dots \le \frac{a-b+2}{2} \le \frac{a-b+1}{1}
    $$ 
    So $$
    C^{b}_{a} = \frac{a}{b}\frac{a-1}{b-1}\dots\frac{a-b+2}{2}\frac{a-b+1}{1} \le (a-b+1)^{b}
    $$ 
    This bound is not sharp, but suffices for our purpose. \newline
\end{myproof}


\begin{proposition}
    No sequence of finite abelian groups has logarithmic diameter. Therefore, no sequence of abelian groups yields an expander family.
\end{proposition}
\begin{remark}
    If a sequence of finite groups admits an unbounded sequence of abelian groups as quotients, then by Quotients Nonexpansion Principle, it does not yield an expander family.
\end{remark}
\begin{myproof}
    \begin{itemize}
        \item[]
        \item Finite abelian group $G$; $\Gamma$\syin$G$; $d=\abs{\Gamma}$; $\gamma_{1}, \dots, \gamma_{d} $ be the elements of $\Gamma$. Let $X=\Cay{G,\Gamma}$, $k=\diam{X}$ Since $\Gamma$ generates $G$, $k$ is finite, and every element of $G$ can be expressed as a word in $\Gamma$ of length less than $k$.
        \item Since $G$ is abelian, we can rearrange the elements in the word, so that every element of $G$ is of the form $$
        e^{a_{0}}\gamma^{a_{1}}_{1}\dots\gamma^{a_{d}}_{d}
        $$
        where $e$ is the identity and $\sum^{d}_{i=0}a_{i}=k $, each $a_{i} $ non-negative integer. (without identity, the sum of $a_{i} $ would not be constant)
        \item By Lemma 4.11, the number of distinct elements of this form is bounded above by $C^{k}_{k+d} $. Then, by Lemma 4.12, $\abs{X} \le C^{k}_{k+d} = C^{d}_{k+d} \le (k+1)^{d} $, so $\diam{X} \ge \abs{X}^{1/d}-1 $
        \item For $\{X_{n}\} $, we have $\diam{X_{n}}\ge \abs{X_{n}}^{1/d}-1 $, but RHS is a roof function, which grows faster than logarithmic functions, done.
        \item[] 
    \end{itemize}
\end{myproof}


\section{Diameters of Subgroups and Quotients}
Diameter's version of Subgroup Nonexpansion Principle, where we connect subgroup and original with spanning subgraph. \newline

\begin{definition}
    Graph $X,Y$. Define composite graph of X and Y, $C(X \times Y)$, as follows.
    \begin{itemize}
        \item Vertex set: $X\times Y$
        \item Set of edges between $(x_{1}, y_{1}) $ and $(x_{2}, y_{2}) $ is the set of pairs $(e_{1},e_{2}) $, s.t. $e_{1} $ is an edge in $X$ between $x_{1}$ and $x_{2} $, and $e_{2} $ is an edge in $Y$ between $y_{1}$ and $y_{2} $. 
        \item[] 
    \end{itemize}
\end{definition}

\begin{example}
See figure 4.1.
    \begin{figure}[htbp!]
        \includegraphics[width=1\linewidth]{Screenshot 2024-09-26 at 19.28.05.png}
        \caption{}
    \end{figure}
    \newline
\end{example}

\begin{definition}
    Graph $X$ with vertex set $V$, edge multiset $E$. A spanning subgraph $X'$ of $X$ is a graph with vertex set $V$ and edge set $E'$, where $E' \subset E$.
\end{definition}
\begin{remark}
    Spanning refer to the fact that $X'$ uses every vertex of $X$. Strangely, $X'$ defined in this way must be a simple graph. Also, it's possible that $X$ is connected while $X'$ is not. \newline
\end{remark}


\begin{lemma}
    Suppose $X$ is a spanning subgraph of a finite graph $Y$. Then $\diam{X}\ge \diam{Y}$.
\end{lemma}
\begin{myproof}
    Some walk in $Y$ may not exist in $X$, but any walk in $X$ must exist in $Y$. So the distance between two vertices in $X$ is no shorter than that in $Y$. \newline
\end{myproof}


\begin{lemma}
    Finite graphs $X,Y$. Then $\diam{C(X\times Y)} \ge \diam{X}$, and $\diam{C(X\times Y)} \ge \diam{Y}$
\end{lemma}
\begin{myproof}
    Let $x_{1}, x_{2} \in X, \; y_{1}, y_{2} \in Y $. A walk of length $l$ in $C(X \times Y)$ from $(x_{1}, y_{1}) $ to $(x_{2}, y_{2}) $ corresponds to a walk of length $l$ in $X$ from $x_{1} $ to $x_{2} $. So distance in $C(X \times Y)$ is no more than that in $X$. \newline 
\end{myproof}


\begin{lemma}
    Finite group $G$, $H \le G$, $T$ is the set of transversals, $\Gamma$\syin$G$. Then $\Cay{G,\Gamma}$ is isomorphic to a spanning subgraph of $C(\Cay{H,\hat{\Gamma}} \times \Cos{H\backslash G, \Gamma })$.
\end{lemma}
\begin{myproof}
    First, we identify vertices. 
    \begin{itemize}
        \item Define $\phi : \, G \rightarrow H \times (H\backslash G) $, by $\phi(g) = (g(\overline{g})^{-1}, Hg) $
        \item For $(h, Ha)$, we can find $\phi(h\overline{a}) = (h, Ha)$. Surjectivity done.
        \item Suppose $ (g_{1}(\overline{g_{1}})^{-1}, Hg_{1}) = (g_{2}(\overline{g_{2}})^{-1}, Hg_{2})  $, then $Hg_{1} = Hg_{2} $, then $\overline{g_{1}} = \overline{g_{2}}$. Since $g_{1}(\overline{g_{1}})^{-1} = g_{2}(\overline{g_{2}})^{-1}$, thus $g_{1} = g_{2} $. Injectivity done.
    \end{itemize}
    Then, we identity edges. Note this is just isomorphic to one spanning subgraph, so we only need to show injectivity.
    \begin{itemize}
        \item $\gamma \in \Gamma, g\in G$. Then $\gamma$ induces an edge in $\Cay{G,\Gamma}$ from $g$ to $g\gamma$. The corresponding edge in $C(\Cay{H,\hat{\Gamma}} \times \Cos{H\backslash G, \Gamma })$ comes from the edge pair $(e_{1}, e_{2})$. 
        \item Suppose we have two equal edge pairs $(e_{1}, e_{2})$ and $(e'_{1}, e'_{2})$, then $e_{1}, e'_{1} $ must be induced from same vertex, $e_{2}, e'_{2} $ must be induced from same vertex, and by arguments in matching vertices, we know the corresponding edges in $\Cay{G,\Gamma}$ must be induced from same vertex, so these two edges are the same. Injectivity done.
        \item[] 
    \end{itemize}
\end{myproof}


\begin{proposition}
    Let $G,H,\Gamma,T$ be defined as above. Then 
    \begin{align*}
        \diam{\Cay{G,\Gamma}} &\ge \diam{\Cay{H,\hat{\Gamma}}} \\
        \diam{\Cay{G,\Gamma}} &\ge \diam{\Cos{H\backslash G, \Gamma}}
    \end{align*}
\end{proposition}
\begin{myproof}
    $\Cay{G,\Gamma}$ - some spanning subgraph of $C(\Cay{H,\hat{\Gamma}} \times \Cos{H\backslash G, \Gamma })$.\newline
\end{myproof}


\begin{proposition}
    $\{G_{n}\} $ a sequence of finite groups. Suppose $\{G_{n}\} $ admits $\{H_{n}\} $ as a bounded-index sequence of subgroups. If $\{H_{n}\} $ does not have logarithmic diameter, so does $\{G_{n}\} $. Therefore, $\{G_{n}\} $ would not yield an expander family.
\end{proposition}
\begin{remark}
    The Quotient Version is false. It is possible that while quotients do not have logarithmic diameter, the originals have. An example is given in section 7.
\end{remark}
\begin{myproof}
    \begin{itemize}
        \item[]
        \item Suppose $\{\Cay{G_{n}, \Gamma_{n}} \} $ has logarithmic diameter for some sets $\Gamma_{n} $, s.t. $\abs{\Gamma_{n}}$ constant and $\Gamma_{n} $\syin$G_{n} $ for all $n$. Let $T_{n} $ be a set of transversals for $H_{n} $ in $G_{n} $. 
        \item Let $M$ s.t. $[G_{n}:H_{n}]\le M $ for all $n$. Let $$
        \Lambda_{n} = \hat{\Gamma_{n}} \cup \{(M-[G_{n}:H_{n}])\abs{\Gamma_{n}}\cdot e_{n}\}
        $$
        \item Following similar argument in Prop.2.24 and by Prop.4.19, we have 
        \begin{align*}
            \diam{\Cay{H_{n},\Lambda_{n}}} &= \diam{\Cay{H_{n},\hat{\Gamma_{n}}}} \\
            &\le \diam{\Cay{G_{n},\Gamma_{n}}} \\
            &\le C\ln\abs{G_{n}} \;\;\; \textrm{(Lemma 4.3)} \\
            &\le C\ln\abs{H_{n}} + C\ln M \;\;\; (\abs{G_{n}} = \abs{H_{n}}[G_{n}:H_{n}]) \\
            &\le 2C'\ln\abs{H_{n}} \;\;\; (C\ln M \, \textrm{must be bounded})
        \end{align*}
        So $\{H_{n}\} $ has logarithmic diameter now, contradiction.
        \item[] 
    \end{itemize}
\end{myproof}


\begin{lemma}
    Dihedral groups $D_{n} $ do not have logarithmic diameter. 
\end{lemma}
\begin{myproof}
    Let $H_{n} = \inner{r} \cong \Z_{n} $, thus $\{H_{n}\} $ do not have logarithmic diameter by Prop.4.13. Note $[D_{n}:H_{n}] =2 $ for all $n$, so by Prop.4.20, $D_{n} $ do not have logarithmic diameter. \newline
\end{myproof}


\section{Solvable Groups with Bounded Derived Length}
\begin{definition}
    Group $G$. An element of the form $a^{-1}b^{-1}ab $ for some $a,b\in G$ is a \define{commutator}. Define $G'$ to be the subgroup of $G$ generated by the set of all commutators in $G$. $G'$ is the \define{commutator subgroup} of $G$. \newline
\end{definition}


\begin{definition}
    Group $G$. We recursively define a sequence of subgroups of $G$, as follows: 
    \begin{align*}
        G^{(0)} &= G, \\
        G^{(1)} &= G', \\
        &\dots \\
        G^{(k+1)} &= (G^{k})'
    \end{align*}
    The group $G^{(k)} $ is the \define{kth derived subgroup} of $G$.
\end{definition}
\begin{remark}
    That is, each recursion is taking a commutator subgroup, and the 1st derived subgroup of $G$ is exactly the commutator subgroup of $G$. \newline
\end{remark}


\begin{definition}
    Group $G$. $G$ is solvable with derived length 0 if $G$ is the trivial group. $G$ is solvable with derived length $k+1$ if $G^{k} $ is nontrivial but $G^{k+1} $ is trivial.
\end{definition}
\begin{remark}
\begin{itemize}
    \item[]
    \item $G$ is abelian iff $G$ is solvable with derived length 1. In this case, the only commutator is the identity.
    \item To say a finite group is solvable means it is "built up out of abelian pieces". The derived length is the minimum number of pieces required.
    \item[] 
\end{itemize}
\end{remark}

\begin{lemma}
    Group $G$. Then:
    \begin{itemize}
        \item[1] $G' \triangleleft G $
        \item[2] If $N$ is a normal subgroup of $G$, then $G/N$ is abelian iff $G'\le N$. That is, $G'$ is the smallest normal subgroup with associated quotient being abelian.
    \end{itemize}
\end{lemma}
\begin{myproof}
    \begin{itemize}
        \item[]
        \item[1] For arbitrary $h \in G'$, $h=a^{-1}b^{-1}ab $ for $a,b\in G$. Then for any $g \in G$, $g^{-1}hg = (g^{-1}a^{-1}b^{-1})(abg) \in G' $, done.
        \item[2] 
        \begin{itemize}
            \item[$(\Rightarrow)$]
            \begin{align*}
                 G/N \; \textrm{abelian}
                &\Leftrightarrow abN = baN, \, \forall a,b\in G \\
                &\Leftrightarrow a^{-1}b^{-1}ab \in N \, \forall a,b\in G \\
                &\Rightarrow G' \subset N
            \end{align*}
            \item[$(\Leftarrow)$]
            \begin{align*}
                 G'\le N 
                &\Rightarrow G' \subset N \\
                &\Rightarrow a^{-1}b^{-1}ab \in N \, \forall a,b\in G \\
                &\Leftrightarrow G/N \; \textrm{abelian}
            \end{align*}
            \item[] 
        \end{itemize} 
    \end{itemize}
\end{myproof}


\begin{proposition}
    $\{G_{n}\} $ is a sequence of finite nontrivial groups s.t. $\abs{G_{n}}\rightarrow \infty $. Let $k$ be a positive integer. Suppose that for all $n$, $G_{n} $ is solvable with derived length $\le k$, then $\{G_{n}\} $ does not yield an expander family.
\end{proposition}
\begin{myproof}
    Proof by induction. $k=1$ abelian group, obvious. Suppose true for $k$, consider $k+1$.
    \begin{itemize}
        \item[\textbf{Case 1}] The sequence $\{G'_{n}\} $ has bounded index in $\{G_{n}\} $. 
        \begin{itemize}
            \item Note $G'_{n} $ does not yield an expander family by induction hypothesis. 
            \item Also note bounded index. Then by Subgroup Nonexpansion Principle, done. 
        \end{itemize}
        \item[\textbf{Case 2}] The sequence $\abs{G_{n}/G'_{n}}$ is unbounded (i.e. index unbounded).
        \begin{itemize}
            \item By Lemma 4.25, $G_{n}/G'_{n} $ always abelian, so it does not yield an expander family. 
            \item This is an unbounded sequence of quotients. Then by Quotients Nonexpansion Principle, done.
        \end{itemize}
        \item[] 
    \end{itemize}
\end{myproof}


\section{Semidirect Product \& Wreath Product}
Wreath product is a special case of the semidirect product, and is essential in the construction of expander families. \newline


\begin{definition}
    Groups $G,K$. Homomorphism $\theta: \, K \rightarrow \Aut{G} $. Define a binary operation $\star$ on $G\times K$ by $$
    (g_{1}, k_{1}) \star (g_{2}, k_{2}) = (g_{1}\theta_{k_1}(g_{2}), k_{1}k_{2})
    $$
    The set $G\times K$ together with $\star$, is called the semidirect product group of $G$ and $K$ with respect to $\theta$, and is denoted as $G\rtimes K$ when the context of $\theta$ is unambiguous. 
\end{definition}
\begin{remark}
    \begin{itemize}
        \item[]
        \item Structure of this construction: $$
        \begin{array}{cc}
            \begin{array}{rcl}
                \theta: \, K  & \rightarrow & \Aut{G} \\
                k & \mapsto & \theta_{k}
            \end{array} & \begin{array}{c}
                \textrm{Special Case:} \\
                 e_{K} \mapsto id
            \end{array}  \\
            & \\
             \begin{array}{rcl}
                \theta_{k} : \, G &\rightarrow & G \\
                g &\mapsto & \theta_{k}(g)
            \end{array} & \begin{array}{rcl}
                id: G &\rightarrow & G \\
                g &\mapsto & g
            \end{array}
        \end{array}
        $$
        \item Identity: $(e_{G}, e_{K}) $
        \item Inverse of $(g,k)$: $(\theta^{-1}_{k}(g^{-1}), k^{-1}) $
        \item Further simplification on notation: $$
        \begin{array}{rcl}
            gk & \textrm{for} & (g, k) \\
            g_{1}k_{1}g_{2}k_{2} & \textrm{for} & (g_{1},k_{1})\star(g_{2},k_{2}) \\
            \leftindex^k {g} & \textrm{for} & \theta_{k}(g) \\
        \end{array}
        $$
        Note $\leftindex^{{k_{1}}k_{{2}}} {g} = \leftindex^{k_{1}} {(\leftindex^{k_{2}} {g})} $, due to homomorphism of $\theta$
        \item When there is no ambigurity, we view $G$ as a subgroup of $G\rtimes K$ by identifying $g\in G$ as $(g,e{G})$. Justification: for any group $G$, we always have $G \cong G \times \{e_{G}\}$. Similarly, view $K$ as a subgroup of $G \rtimes K$ by identifying $k\in K$ as $(e_{K}, k) $. 
        \item In this book's context, we are implicitly treating $G$ and $K$ as subgroups of a larger group, so $e_{K} = e_{G} = 1 $.
        \item[] 
    \end{itemize}
\end{remark}


\begin{lemma}
    $G \triangleleft G\rtimes K $, and $(G \rtimes K)/G \approx K $.
\end{lemma}
\begin{remark}
    Complete notation: $G \times \{e_{G}\} \triangleleft G\rtimes K $, and $(G \rtimes K)/G \cong K \times \{e_{K}\} $ \newline \textcolor{red}{Caution:} These results built on treating $e_{K} = e_{G} = 1 $, i.e. $G$ and $K$ are subgroups of a larger group.
\end{remark}
\begin{myproof}
    \begin{itemize}
        \item[]
        \item[1] For arbitrary $(t,1)$ in $G$, for any $(g,k) \in G\rtimes K $, 
        \begin{align*}
            (g,k)\star(t,1)\star((g,k))^{-1} &= (g\theta_{k}(t), k)(\theta^{-1}_{k}(g^{-1}),k^{-1}) \\
            &= (g\theta_{k}(t)g^{-1},1) \in G   
        \end{align*}
        Normality done.
        \item[2] LHS is the left coset of $G$ in $G \rtimes K$, $gkG=\{(g\theta_{k}(t),k) \, \vert \, t\in G\}$. Note $\theta_{k} $ is an automorphism of $G$ (isomorphism from $G$ to $G$), thus for arbitrary $g_{1}, g_{2} \in G $, $g_{1}kG = g_{2}kG $. That is, $gkG$ is determined by $k$ solely. The isomorphism is obvious.
        \item[] 
    \end{itemize}
\end{myproof}


\begin{definition}
\begin{itemize}
    \item[] 
    \item Let $I$ be a finite (index) set, and $G,K$ be groups. Let $G^{I} = \oplus_{i\in I}G $ be the direct product of several copies of $G$, one for each element of $I$. Elements of $G$ are $\abs{I}$-tuples $(g_{i})_{i\in I} $, where $g_{i} \in G $ for all $i$. 
    \item Let $\theta$ be a homomorphism from $K$ to $S^{I} $, where $S^{I} $ is the symmetric group on $I$ (set of all permutations of $I$). Later, call this as "defining $\theta$ as an action of $K$ on $I$".
    \item Then $\theta$ induces a homomorphism, denoted as $\theta: \, K \rightarrow \Aut{G^{I}} $, by $k \mapsto \theta_{k} $, where $\theta_{k}$ is the permutation in $S^{I} $ which corresponds to $k$.
    \item[] 
\end{itemize}
\end{definition}


\begin{definition}
    The wreath product of $G$ and $K$ with respect to $\theta$ and $I$ is denoted $G\wr K$ and defined as $$G\wr K = G^{I}\rtimes K $$
\end{definition}
\begin{remark}
    \begin{itemize}
        \item[]
        \item Wreath product is a special case of semidirect product, thus preserving all its properties. That is, $G\wr K$ is the wreath product group, $G^{I} \triangleleft G\wr K$, and $(G\wr K)/G^{I} \approx K$.
        \item Structure of this construction:
        $$
        \begin{array}{cc}
            \begin{array}{rcl}
                \theta: \, K  & \rightarrow & \Aut{G} \\
                k & \mapsto & \theta_{k}
            \end{array} & \begin{array}{c}
                \textrm{Special Case:} \\
                 e_{K} \mapsto id \, (\textrm{no permutation})
            \end{array}  \\
            & \\
             \begin{array}{rcl}
                \theta_{k} : \, G^{I} &\rightarrow & G^{I} \\
                (g_{i})_{i\in I} &\mapsto & (g_{\theta_k (i)})_{i\in I}
            \end{array} & \begin{array}{rcl}
                id: G &\rightarrow & G \\
                (g_{i})_{i\in I} &\mapsto & (g_{i})_{i\in I}
            \end{array}
        \end{array}
        $$
        Note every $\theta_{k}$ is now a permutation on $I$.
        \item Binary operation: $((g^{1}_{i})_{i\in I},x) \star ((g^{2}_{i})_{i\in I},y) = ((g^{1}_{i})_{i\in I}(g^{2}_{\theta_x (i)})_{i\in I},xy)  $
        \item Identity: $((e_{G_i})_{i\in I}, e_{K})$
        \item Inverse of $((g_{i})_{i\in I}, k) $: $((g^{-1}_{\theta^{-1}_k (i)})_{i\in I}, k^{-1}) $, s.t. 
        \begin{align*}
            ((g_{i})_{i\in I}, k) \star ((g^{-1}_{\theta^{-1}_k (i)})_{i\in I}, k^{-1}) &= ((g_{i})_{i\in I}(g^{-1}_{\theta_k\theta^{-1}_k (i)})_{i\in I}, e_{K}) \\
            &= ((g_{i})_{i\in I}(g^{-1}_{i})_{i\in I}, e_{K}) \\
            &= ((e_{G_i})_{i\in I}, e_{K})
        \end{align*}
        \item[] 
    \end{itemize}
\end{remark}


\section{Counterexample: Cube-Connected Cycle Graphs}
This counterexample is motivated by following questions: 
\begin{itemize}
    \item[1] Is there a diameter version for Quotients Nonexpansion Principle?
    \item[2] If a sequence of finite groups has logarithmic diameter, does it necessarily yield an expander family?
    \item[3] Can an unbounded sequence of solvable groups with bounded derived length have logarithmic diameter? (a more concerete question based on 2)
\end{itemize}
The answers are no, no, and yes. The family of cube-connected cycle graphs would be the example. \newline

We make following conventions on notation. Consider positive integer $n$.
\begin{itemize}
    \item $e_{i} $ denote the element of $\Z^{n}_{2} = \Z_{2} \times \cdots \times \Z_{2} $, wiht 1 in $i$th place and 0 elsewhere. 
    \item $0 = (0, \cdots, 0) $ denotes the identity of $\Z^{n}_{2} $
    \item[] 
\end{itemize}


\begin{definition}
\begin{itemize}
    \item[]
    \item Define an action $\theta$ of $\Z_{n} $ on $I = \Z_{n} $ by $\theta_{a}(b) = a+b $. Via this action, construct the wreath product group $G_{n} = \Z_{2} \wr \Z_{n} $
    \item Let $\Gamma_{n} = \{(e_{n},0),\gamma,\gamma^{-1}\} \subset G_{n} $, where $\gamma =(0,1), \, \gamma^{-1}=(0,-1) $.
    \item Define the cube-connected cycle graph $CCC_{n} $ be the Cayley graph $\Cay{G_{n}, \Gamma_{n}}$.
\end{itemize}
\end{definition}
\begin{remark}
\begin{itemize}
    \item[]
    \item $G_{n} = \Z^{n}_{2}\rtimes Z_{n} $, the action of $\Z_{n} $ on $\Z^{2}_{n} $ is given by 
    \begin{align*}
        \leftindex^{1} (a_{1},a_{2},\dots,a_{n}) &= (a_{n},a_{1},\dots,a_{n-1}) \\
        \leftindex^{k} (a_{1},a_{2},\dots,a_{n}) &= (a_{n-k+1},a_{n-k+2},\dots,a_{n-k})
    \end{align*}
    Since 1 is the generator of $\Z_{n} $. Note the identity is 0.
    \item Structure of this construction:
            $$
        \begin{array}{cc}
            \begin{array}{rcl}
                \theta: \, \Z_{n}  & \rightarrow & \Aut{\Z^{\Z_{n}} } \\
                a & \mapsto & \theta_{a}
            \end{array} & \begin{array}{c}
                \textrm{Special Case:} \\
                 0 \mapsto id \, (\textrm{no permutation})
            \end{array}  \\
            & \\
             \begin{array}{rcl}
                \theta_{a} : \, \Z_{n}^{\Z_{n}} &\rightarrow & \Z_{n}^{\Z_{n}} \\
                (g_{i})_{i\in \Z_{n}} &\mapsto & (g_{i+n-a})_{i\in \Z_{n} }
            \end{array} & \begin{array}{rcl}
                id: \Z_{n}^{\Z_{n}} &\rightarrow & \Z_{n}^{\Z_{n}} \\
                (g_{i})_{i\in \Z_{n}} &\mapsto & (g_{i})_{i\in \Z_{n}}
            \end{array}
        \end{array}
        $$
    \item Note $\Gamma_{n} $ is symmetric, thus $CCC_{n} $ is undirected. Also, $\abs{\Gamma}=3$, so $CCC_{n} $ is 3-regular.
    \item[] 
\end{itemize}
\end{remark}


\begin{example}
    Let $n=3$. For simplification, denote elements of $\Z^{3}_{2} $ as binary strings (and follow this convention later). Then, $(e_{3},0) = (001,0); \; \gamma = (000,1); \; \gamma^{-1} = (000,-1) $. \newline
    Consider (100,1). Then 
    $$
    \begin{array}{rcccl}
        (100,1)(000,1) &=& (100+000,1+1) &=& (100,2) \\
        (100,1)(000,-1) &=& (100+000,1-1) &=& (100,0) \\
        (100,1)(001,0) &=& (100+100,1+0) &=& (000,1) \\
    \end{array}
    $$
    Hence (100,1) is adjacent to (100,2),(100,0), and (000,1) in $CCC_{3} $.
\end{example}
\begin{remark}
    We express elements of $\Z^{n}_{2} $ as strings of $n$ binary digits. An $n$-dimensional hypercube is the graph whose vertices are the elements of $\Z_{2}^{n} $, where two vertices are adjacent, via an edge of multiplicity one, if they differ in exactly one digit, and they are nonadjacent otherwise. \newline
    
    It turns out that $CCC_{n} $ can be visualised as a $n$-dimensional hypercube, where each vertex is replaced by an $n$-cycle composed of elements of $\Z_{n}$ See Figure 4.2. No proof for this visualisation, since it is not the purpose.
\end{remark}

\begin{figure}[htbp!]
    \includegraphics[width=1\linewidth]{Screenshot 2024-09-27 at 21.13.08.png}
    \caption{$CCC_{3} $}
\end{figure}


\begin{proposition}
    For all $n$, $\diam{CCC_{n}}\le 4n$.
\end{proposition}
\begin{myproof}
    \begin{itemize}
        \item[]
        \item An arbitrary element of $G_{n} $ is of the form $(e_{j_1}e_{j_2}\cdots e_{j_k},a)$ for some positive integers $j_{1},\dots,j_{k} $ with $1\le j_{1} < j_{2} < \cdots < j_{k} \le n $ and $k\le n$ (i.e. n bites, either 1 or 0). 
        \item By Prop.4.10, suffice to show the word norm of $(e_{j_1}e_{j_2}\cdots e_{j_k},a) \le 4n $. 
        \item Let $e=(e_{n},0) $. Note for all positive integers $c$, 
        \begin{align*}
        \gamma^{c}e(\gamma^{-1})^{c} &= (0,c)(e_{n},0)(0,-c) \\
        &= (0+\leftindex^{c} e_{n}, c+0)(0,-c) \\
        &= (\leftindex^{c} e_{n}, c)(0,-c) \\
        &= (\leftindex^{c} e_{n}, 0) \; \; \; \textrm{0 always mapped to identity 0} \\
        &= (e_{c},0)
        \end{align*}
        \item Also note, $(g_{1},0)(g_{2},0) = (g_{1}+\theta_{0}(g_{2}),0) = (g_{1}+g_{2},0)  $, since 0 is the identity.
        \item Then 
        \begin{align*}
        (e_{j_1}e_{j_2}\cdots e_{j_k},a) 
        &= (e_{j_1},0)(e_{j_2},0)\dots(e_{j_k},0)(0,a) \\
        &= \gamma^{j_{1}}e\gamma^{-j_{1}}\gamma^{j_{2}}e\gamma^{-j_{2}}\dots\gamma^{j_{k}}e\gamma^{-j_{k}}\gamma^{a} \\
        &= \gamma^{j_{1}}e\gamma^{j_{2}-j_{1}}e\gamma^{j_{3} -j_{2}}\dots\gamma^{j_{k}-j_{k-1}}e(\gamma^{-1})^{j_{k}}\gamma^{a} \\
        \end{align*}
        Note $\gamma$ appears $j_{1} + (j_{2}-j_{1})+\dots + (j_{k}-j_{k-1})+a = j_{k}+a $ times, $\gamma^{-1}$ appears $j_{k} $ times, and $e$ appears $k$ times. So the word norm of $(e_{j_1}e_{j_2}\cdots e_{j_k},a)$ has $2j_{k} + a + k \le 4n $
        \item[] 
    \end{itemize}
\end{myproof}


\begin{lemma}
    The sequence $\{CCC_{n}\} $ has logarithmic diamter. 
\end{lemma}
\begin{myproof}
    $\abs{G_{n}}=\abs{\Z^{n}_{2}}\abs{\Z_{n}}=n2^{n} $, thus $\abs{CCC_{n}}=\ln n+n\ln 2$. Let $C=4/\ln 2$, by Prop.4.4, $$
    \diam{CCC_{n}} \le 4n \le C(\ln n+n\ln 2) = C\ln\abs{CCC_{n}} 
    $$ 
    \begin{itemize}
        \item[] 
    \end{itemize}
\end{myproof}


\begin{lemma}
    $G_{n} $ is solvable with derived length 2. Therefore, the sequence $\{CCC_{n} \} $ is not an expander family. 
\end{lemma}
\begin{myproof}
    Note $G_{n} $ is not abelian as $\gamma e_{n} \ne e_{n}\gamma $, so $G_{n} $ does not have derived length 1. By Lemma 4.28, $\Z^{n}_{2} \triangleleft G_{n} $, and by Lemma 4.25 $G'_{n} $ is the smallest normal subgroup that is abelian, thus $G'_{n} \le \Z^{n}_{2} $. Since $\Z^{n}_{2} $ is the direct product of $Z_{2} $ (cyclic group of prime order), by fundamental theorem of abelian groups, $\Z^{n}_{2} $ is abelian, so is $G'_{n} $. Then, $G^{(2)}_{n} = 1 $.
\end{myproof}
\begin{remark}
    Now we can answer questions asked in the beginning. 
    \begin{itemize}
        \item[1] In the proof above, we note $\Z^{n}_{2} \triangleleft G_{n}  $ and $G_{n}/\Z^{n}_{2} \cong \Z_{n} $, which is a sequence of quotients. Also, as an abelian group, $\Z_{n} $ does not have logarithmic diameter. However, by Lemma 4.33, $G_{n} $ does have logarithmic diameter. 
        \item[2] By Lemma 4.33 and Lemma 4.34.
        \item[3] By Lemma 4.33 and Lemma 4.34.
    \end{itemize}
\end{remark}



\end{document}
